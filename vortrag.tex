\documentclass{beamer} 
%\documentclass[handout]{beamer} 

% Michael Maier, 2015.
% CC-0

\usepackage[utf8]{inputenc}
\usepackage[ngerman]{babel}

\title{OpenStreetMap und Wikidata} 
\author{Michael Maier \textless Michael.Maier@mailbox.org\textgreater} 
\date{4. Juli 2016} 

\usetheme{Antibes}

\newcommand{\boldm}[1] {\mathversion{bold}#1\mathversion{normal}}

\hypersetup{colorlinks=true,urlcolor=blue,linkcolor=white}

%\usebackgroundtemplatei{
%\includegraphics[width=\paperwidth,
%height=0.8\paperheight]{mag_map.png}
%}

\begin{document}

%\maketitle

\begin{frame} 


\begin{figure}
  \centering
  \includegraphics[width=.5\textwidth]{mag_map.png}
\end{figure}

\begin{center}
\Huge{OpenStreetMap und Wikidata\\}
\end{center}

\begin{center}
\Large{\emph{auf dem Weg zu Linked Open Geodata}}
\end{center}

\end{frame}


\section{Einleitung}


\begin{frame}{Vorstellung}

  \begin{itemize}
    \item Michael Maier \textless \href{mailto:Michael.Maier@mailbox.org}{Michael.Maier@mailbox.org}\textgreater
    \item Student an der TU Graz (Telematik)
\vspace{0.3cm}
    \item Linux-User (Debian/grml) seit 2004
    \item Organisiere Grazer Linuxtage seit 2011 mit
    \item OpenStreetMap als Hobby seit Juli 2010
    \item Leite den Grazer OSM-Stammtisch seit Mai 2011
\vspace{0.3cm}
    \item Vorträge und Workshops zum Thema OSM seit 2012
    \item Freiberuflich OSM-Aufträge und Consulting
    \begin{itemize}
      \item OSM-username: \emph{\href{http://www.openstreetmap.org/user/species}{species}}
      \item Github-Account: \emph{\href{https://github.com/species}{species}}
      \item Twitter-Account: \emph{\href{https://twitter.com/osmgraz}{@osmgraz}}
    \end{itemize}
  \end{itemize}
\end{frame}


\begin{frame}{Linked Open Data}

  Tim-Berners Lee schlägt ein Sterne-Schema für Open Data vor:
 \vspace*{0.4cm}

  \url{http://5stardata.info}
 \vspace*{0.4cm}

  \begin{itemize}
    \item[] \textcolor{orange}{\boldm$\star$} make it available on the web under an open license \pause \includegraphics[width=.3cm]{check.png} ODbL \pause
    \item[] \textcolor{orange}{\boldm$\star$$\star$} make it available as structured data (Karten nicht JPG) \pause  \includegraphics[width=.3cm]{check.png} \pause
    \item[] \textcolor{orange}{\boldm$\star$$\star$$\star$} make it available in a non-proprietary open format \pause  \includegraphics[width=.3cm]{check.png} \pause
    \item[] \textcolor{orange}{\boldm$\star$$\star$$\star$$\star$} use URIs to denote things \pause  \includegraphics[width=.3cm]{check-yellow.png} http://osm.org/node/1 \pause
    \begin{itemize}
      \item[] \hspace{0.5cm}Use RDF ... ? \pause $\Rightarrow$ \url{http://linkedgeodata.org} \includegraphics[width=.3cm]{check-yellow.png} \pause
    \end{itemize}
    \item[] \textcolor{orange}{\boldm$\star$$\star$$\star$$\star$$\star$} link your data to other data \dots \pause 1. Schritt: Wikidata!
  \end{itemize}


\end{frame}

\section{Wikidata}

\begin{frame}{Was ist Wikidata}

Ein Projekt der Wikimedia Foundation

\begin{itemize}
  \item Wikidata ist eine freie Wissensdatenbank

    Release 2013 von der Wikimedia Deutschland


    CC0 1.0



    \begin{itemize}
      \item \emph{Eigentlich eine Geo-Datenbank}
    \end{itemize}
\pause
  \item Entsteht aus der Arbeit von \textgreater 2,5\,M Hobbykartografen "`\emph{Mapper}"'

 \item Das komplette "`planet file"' ist ca. 74\,GB groß (xml.bz2) (Mittwoch):
  \begin{itemize}
    \item 3.297.315.809 Nodes
    \item 339.736.038 Ways
    \item 4.126.808 Relations
  \end{itemize}


\end{itemize}

\end{frame}

\begin{frame}{Warum OpenStreetMap?}

\end{frame}


\begin{frame}{Wer steht hinter OpenStreetMap}


\end{frame}

\section{Wie funktioniert OpenStreetMap?}

\subsection{Technologie}

\section{OpenStreetMap Nutzen}

\subsection{Rohdaten}

\begin{frame}{Hilfe}
Fragen? 
\begin{itemize}
  \item Dokumentation: \href{http://wiki.openstreetmap.org}{wiki.openstreetmap.org}
  \begin{itemize} 
    \item Mitmachen? \href{http://learnosm.org/}{learnosm.org}
  \end{itemize}
  \item Immer noch etwas unklar? $\Rightarrow$ Mailingliste \href{http://lists.openstreetmap.org/listinfo/talk-at}{talk-at}
 \vspace*{0.4cm}
  \item Weltweite \href{http://usergroups.openstreetmap.de/}{Stammtische}
  \begin{itemize}
    \item 1/Monat Graz
    \item 1/Monat Wien
    \item 1/Monat Innsbruck
  \end{itemize}
 \vspace*{0.4cm}
  \item Grazer Linuxtage, 29.-30. April

\end{itemize}

 \vspace*{-2.8cm}

\begin{itemize}
  \item Konferenz: \href{http://stateofthemap.org/}{State of the Map}, 23.-25. September, Brüssel
\end{itemize}
\end{frame}

\section{Ende}

\begin{frame}{Vielen Dank für die Aufmerksamkeit!}

  Folien zur FOSSGIS 2016, 4.7.2016, Salzburg
\vspace{1cm}

Erstellt mittels \LaTeX Beamer, Quelltext: \href{https://github.com/species/vortrag-osm-wikidata-fossgis16}{Github/species/vortrag-osm-wikidata-fossgis16}.
\vspace{1cm}

\href{mailto:michael.maier@mailbox.org}{Michael Maier}

Twitter: \href{https://twitter.com/osmgraz}{@osmgraz}
\vspace{1cm}

Folien unter: \includegraphics[width=1cm]{cc-zero.pdf}. 

Alle Daten ODbL, OpenStreetMap Contributors.

\end{frame}



\end{document}
